\documentclass[journal,12pt,twocolumn]{beamer}
\usetheme{CambridgeUS}
\usepackage[utf8]{inputenc}

\usepackage{tfrupee}
\usepackage{enumitem}
\usepackage{amsmath}
\usepackage{amssymb}
\usepackage{graphicx}

\providecommand{\sbrak}[1]{\ensuremath{{}\left[#1\right]}}
\providecommand{\lsbrak}[1]{\ensuremath{{}\left[#1\right.}}
\providecommand{\rsbrak}[1]{\ensuremath{{}\left.#1\right]}}
\providecommand{\brak}[1]{\ensuremath{\left(#1\right)}}
\providecommand{\lbrak}[1]{\ensuremath{\left(#1\right.}}
\providecommand{\rbrak}[1]{\ensuremath{\left.#1\right)}}
\providecommand{\cbrak}[1]{\ensuremath{\left\{#1\right\}}}
\providecommand{\lcbrak}[1]{\ensuremath{\left\{#1\right.}}
\providecommand{\rcbrak}[1]{\ensuremath{\left.#1\right\}}}



\newcommand{\myvec}[1]{\ensuremath{\begin{pmatrix}#1\end{pmatrix}}}
\let\vec\mathbf

\title{Assignment 4}
\author{Ravula Karthik (AI21BTECH11024)}
\date {May 2022}
\begin{document}
	\begin{frame}
		\titlepage 
	\end{frame}
	
	\begin{frame}{Question : NCERT Class 12 Exercise 13.3 Problem 4}
		
		\textbf{Trains X and Y arrive at a station at random between 8 A.M. and 8.20 A.M. Train
			X stops for four minutes and train Y stops for five minutes. Assuming that the trains
			arrive independently of each other, determine various probabilities related to the times x and y of their respective arrivals? }
	\end{frame}
	\begin{frame}{Solution}
		
	 The event 
	
	A =  X  arrives in the interval  $ (t_1 , t_2) $ 
	\begin{align}
		\implies	&P(A) = \frac{t_2 -t_1}{20} 
	\end{align} 

  The event
 
 B =  Y arrives in the interval  $ (t_3 , t_4) $
 \begin{align}
 	\implies	&P(B) = \frac{t_4 -t_3}{20} 
 \end{align} 
 
 Interpreting the independence of the rival
 times as independence of the events A and B, we obtain ,
 \begin{align}
 	\implies	&P(AB) = P(A)P(B) = \frac{(t_4 -t_3)(t_2 -t_1 )}{400} 
 \end{align} 
	
	\end{frame}
	\begin{frame}
	
			(i) We shall determine the probability that train X arrives before train Y.
			
			This is the probability of the event
			\begin{align}
				&C = \cbrak{x \leq y } \nonumber
			\end{align} 
			This event is a triangle with area 200. Hence
			\begin{align}
				\implies    &P(C) = \frac{200}{400} 
			\end{align}
	\end{frame}
		

	\begin{frame}
	 (ii)	We shall determine the probability that the trains meet at the station. For the
		trains to meet, x must be less than y + 5 and y must be, less than x + 4. This is the event
		\begin{align}
			&D = \cbrak{-4 \leq x-y \leq 5} \nonumber
		\end{align} 
		The region D consists of two trapezoids with common base, and its area equals 159.5. Hence 
		\begin{align}
			\implies    &P(D) = \frac{159.5}{400} 
		\end{align}
	\end{frame}
 \begin{frame}
 	(iii) Assuming that the trains met, we shall determine the probability that train X
 	arrived before train Y. We wish to find the conditional probability P(C/D). The event
 	CD is a trapezoid and its area equals 72. Hence
 	
 	\begin{align}
 		&P(C/D) = \frac{P(CD)}{P(D)} =\frac{72}{159.5}
 	\end{align}  
\end{frame}
	
\end{document}