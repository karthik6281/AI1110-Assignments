\documentclass[journal,12pt,twocolumn]{beamer}
\usetheme{CambridgeUS}
\usepackage[utf8]{inputenc}

\usepackage{tfrupee}
\usepackage{enumitem}
\usepackage{amsmath}
\usepackage{amssymb}
\usepackage{graphicx}

\providecommand{\sbrak}[1]{\ensuremath{{}\left[#1\right]}}
\providecommand{\lsbrak}[1]{\ensuremath{{}\left[#1\right.}}
\providecommand{\rsbrak}[1]{\ensuremath{{}\left.#1\right]}}
\providecommand{\brak}[1]{\ensuremath{\left(#1\right)}}
\providecommand{\lbrak}[1]{\ensuremath{\left(#1\right.}}
\providecommand{\rbrak}[1]{\ensuremath{\left.#1\right)}}
\providecommand{\cbrak}[1]{\ensuremath{\left\{#1\right\}}}
\providecommand{\lcbrak}[1]{\ensuremath{\left\{#1\right.}}
\providecommand{\rcbrak}[1]{\ensuremath{\left.#1\right\}}}



\newcommand{\myvec}[1]{\ensuremath{\begin{pmatrix}#1\end{pmatrix}}}
\let\vec\mathbf

\title{Assignment 7}
\author{Ravula Karthik (AI21BTECH11024)}
\date {May 2022}
\begin{document}
	\begin{frame}
		\titlepage 
	\end{frame}
	
	\begin{frame}{Question : EX 5.39}
		
		A random variable x has a geometric distribution if
		
	\centerline{P\brak{x=k} = $ pqk$ $k = 0,1,...$ $ p + q = 1 $ }
	
Find {\ $\Gamma $\brak{z} } and show that \ $ \eta$ $_x$\  = $\frac{q}{p}$ , \ $\sigma$$_y ^2$\  = $\frac{q}{p^2}$
		
	\end{frame}
	\begin{frame}{Solution}
		  
		We already know
		\begin{equation}
			\label{eq:1}
			p + q = 1
		\end{equation}
		
		\begin{align}
			\Gamma\brak{z}  &= \sum_{k = 0}^{\infty} pq^kz^k\\
			\text{from \eqref{eq:1}}     \nonumber\\
			&= \frac{p}{1-qz}
		\end{align} 
	\end{frame}
	
	\begin{frame}
		
	Now ,
	
	\begin{align}
		\Gamma'\brak{z}  &= \frac{pq}{(1-qz)^2}\\
		\text{and}                  \nonumber\\
		\Gamma''\brak{z}  &= \frac{2pq^2}{(1-qz)^3}
	\end{align} 
	
	And 
	
	\begin{align}
		\Gamma'\brak{1}  &= \frac{pq}{(1-q)^2} = \frac{q}{p} =  \eta _x\  \\
		\text{and}                  \nonumber\\
		\Gamma''\brak{1}  &= \frac{2pq^2}{(1-q)^3} = \frac{2q^2}{p^2} = m_2 - m_1
	\end{align} 

	\end{frame}
	
	
	\begin{frame}
	Then 
	
	\begin{align}
		\sigma^2  &= m_2 - (m_1)^2 \\
		&= \frac{2q^2}{p^2} + m_1 - (m_1)^2  \\
		&= \frac{q}{p^2}
	\end{align} 
	
	$\therefore$  \ $\sigma$$_y ^2$\  = $\frac{q}{p^2}$ and \ $ \eta$ $_x$\ = $\frac{q}{p}$ 
	\end{frame}
	
	
\end{document}