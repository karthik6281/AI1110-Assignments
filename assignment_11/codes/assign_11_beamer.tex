\documentclass[journal,12pt,twocolumn]{beamer}
\usetheme{CambridgeUS}
\usepackage[utf8]{inputenc}

\usepackage{tfrupee}
\usepackage{enumitem}
\usepackage{amsmath}
\usepackage{amssymb}
\usepackage{graphicx}

\providecommand{\sbrak}[1]{\ensuremath{{}\left[#1\right]}}
\providecommand{\lsbrak}[1]{\ensuremath{{}\left[#1\right.}}
\providecommand{\rsbrak}[1]{\ensuremath{{}\left.#1\right]}}
\providecommand{\brak}[1]{\ensuremath{\left(#1\right)}}
\providecommand{\lbrak}[1]{\ensuremath{\left(#1\right.}}
\providecommand{\rbrak}[1]{\ensuremath{\left.#1\right)}}
\providecommand{\cbrak}[1]{\ensuremath{\left\{#1\right\}}}
\providecommand{\lcbrak}[1]{\ensuremath{\left\{#1\right.}}
\providecommand{\rcbrak}[1]{\ensuremath{\left.#1\right\}}}



\newcommand{\myvec}[1]{\ensuremath{\begin{pmatrix}#1\end{pmatrix}}}
\let\vec\mathbf

\title{Assignment 11}
\author{Ravula Karthik (AI21BTECH11024)}
\date {June 2022}
\begin{document}
	\begin{frame}
		\titlepage 
	\end{frame}
	
	\begin{frame}{Question : EX 10.21}
		Given a WSS process $x(t)$ and a set of Poisson points t independent of $x(t)$ and with
		average density $\lambda$. we form the sum
		
		$ X_c\brak{w} = \sum_{|t_i|\leq c} x\brak{t_i} e^{-jwt_i} $
		
		Show that if $E\cbrak{x(t)}$ = 0 and $\int_{-\infty}^{\infty} |{X_c(w)}| d\tau < \infty,$ then for large c 
		$E\cbrak{|{X_c(w)}^2|} = 2cS_x(w) + \frac{2c}{\lambda}R_x(0)$ .
		
	\end{frame}
	\begin{frame}{Solution}
	
	We shall show that if
	\begin{align}
		\underline{X}_c(w) &= \frac{1}{\lambda}\sum_{|{t_i}| \leq c}\underline{x}\brak{t_i} e^{-jwt_i} \nonumber \\
		&= \frac{1}{\lambda}\int_{-a}^{a}\underline{x}\brak{t}\underline{z}\brak{t}e^{-jwt} \nonumber
	\end{align}
	
	where $z(t)$ = $\sum \delta(t-\underline{t_i}) $ is a Poisson impulse train, then
	
	$E\cbrak{|{X_c(w)}^2|} = 2cS_x(w) + \frac{2c}{\lambda}R_x(0)$
	
	\end{frame}
	
	\begin{frame}
	\underline{proof} 
	
	Since $R_z(\tau)$ = $\lambda^2 +\lambda\delta(r) $ , it follows that
	
	\begin{align}
		E\cbrak{|{\underline{X}_c(w)^2}|} &= \frac{1}{\lambda^2}\int_{-c}^{c}\int_{-c}^{-c} R_x(t_1-t_2)e^{-jw(t_1-t_2)}dt_1dt_2 \\
		&= \int_{-c}^{c} e^{jwt_2}\int_{-c}^{-c} R_x(t_1-t_2)e^{-jwt_1}dt_1dt_2 +\frac{1}{\lambda}\int_{-c}^{c}R_x(0)dt_2
	\end{align} 
	
	If $\int_{-\infty}^{\infty}|{R_x(\tau)}| < \infty$ then for sufficient large c , the inner integral on the right is nearly equal to $S_x(w)e^-jwt_2 $ and (i) follows.
	
	\end{frame}
	
	
\end{document}